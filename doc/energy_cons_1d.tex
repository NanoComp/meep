\begin{comment}
#include <stdio.h>
#include <stdlib.h>
#include <signal.h>
\end{comment}

\section{Energy conservation in one dimension}

In this example, we compute the total energy over time for a polaritonic
material in one dimension to verify that it is indeed conserved.  This also
demostrates the support for 1D systems.

\begin{figure}
\label{econs_1d}
\caption{Energy vs. Time.}
\epsfig{file=energy_cons_1d-out/energy.eps,width=8.8cm}
\end{figure}

To begin with, to use a 1D system, we include ``tidod.h'' rather than
``dactyl.h.''  ``tidod'' stands for TIme Domain One Dimensional.  

\begin{verbatim}
#include "tidod.h"

const int rad = 10;
\end{verbatim}

For our example polaritonic material, we'll use an $\epsilon(0)$ of 13.4.

\begin{verbatim}
double eps(double z) { return 13.4; }
double one(double z) { return 1; }
\end{verbatim}
\begin{comment}
int main(int argc, char **argv) {
  deal_with_ctrl_c();
  const double ttot = 600.0;
\end{comment}
In various places we will need to add a ``\verb!_1d!'' as we make the
change from a cylindrical system to a 1D system.
\begin{verbatim}
  mat_1d ma(eps, 10.0, rad);
\end{verbatim}
\begin{comment}
  const char *dirname = make_output_directory(argv[0]);
  ma.set_output_directory(dirname);
\end{comment}
The polarizability is added as usual...
\begin{verbatim}
  ma.add_polarizability(one, 0.25, 0.01, 3.0);
  fields_1d f(&ma);
  grace g("energy", dirname);
\end{verbatim}
We use several plane wave sources, to cover a broad frequency range, just
for the heck of it.
\begin{verbatim}
  f.add_source(Ex, 0.6 , 0.8, 0.0, 8.0, 2.0);
  f.add_source(Ex, 0.4 , 0.8, 0.0, 8.0, 2.0);
  f.add_source(Ex, 0.33, 0.8, 0.0, 8.0, 2.0);
\end{verbatim}
\begin{comment}
  double next_printtime = 10;
  while (f.time() < ttot && !interrupt) {
    if (f.time() >= next_printtime) {
      next_printtime += 10;
      printf("Working on time %lg...  ", f.time());
      printf("energy is %lg\n", f.total_energy());
\end{comment}
We plot the total energy, the electromagnetic energy and the
``thermodynamic energy'' which is the energy that is either stored in the
polarization, or has been converted into heat, or (if we had a saturating
gain system) perhaps is stored in a population inversion.
\begin{verbatim}
      g.output_out_of_order(0, f.time(), f.total_energy());
      g.output_out_of_order(1, f.time(),
                            f.electric_energy_in_box(0.0,10.0)
                          + f.magnetic_energy_in_box(0.0,10.0));
      g.output_out_of_order(2, f.time(),
                            f.thermo_energy_in_box(0.0,10.0));
\end{verbatim}
\begin{comment}
    }
    f.step();
  }
}
\end{comment}

